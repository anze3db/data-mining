% To je predloga za poročila o domačih nalogah pri predmetih, katerih
% nosilec je Blaž Zupan. Seveda lahko tudi dodaš kakšen nov, zanimiv
% in uporaben element, ki ga v tej predlogi (še) ni. Več o LaTeX-u izveš na
% spletu, na primer na http://tobi.oetiker.ch/lshort/lshort.pdf.
%
% To predlogo lahko spremeniš v PDF dokument s pomočjo programa
% pdflatex, ki je del standardne instalacije LaTeX programov.

\documentclass[a4paper,11pt]{article}
\usepackage{a4wide}
\usepackage{fullpage}
\usepackage[utf8x]{inputenc}
\usepackage[slovene]{babel}
\selectlanguage{slovene}
\usepackage[toc,page]{appendix}
\usepackage[pdftex]{graphicx} % za slike
\usepackage{setspace}
\usepackage{color}
\definecolor{light-gray}{gray}{0.95}
\usepackage{listings} % za vključevanje kode
\usepackage{hyperref}
\usepackage{float}
\usepackage{verbatim}
\renewcommand{\baselinestretch}{1.2} % za boljšo berljivost večji razmak
\renewcommand{\appendixpagename}{Priloge}

\lstset{ % nastavitve za izpis kode, sem lahko tudi kaj dodaš/spremeniš
language=Python,
basicstyle=\footnotesize,
basicstyle=\ttfamily\footnotesize\setstretch{1},
backgroundcolor=\color{light-gray},
}

\title{Tretja domača naloga}
\author{Anže Pečar (63060257)}
\date{\today}

\begin{document}

\maketitle

\section{Uvod}

Cilj domače naloge je bil oddati napovedi na tekmovalni stržnik in se seznaniti z ocenjevanjem točnosti in napovedni modeli.

\section{Metode}
\subsection{Ocenjevanje točnosti}
Ocenjevanje točnosti na učnih podatkih, F mera
\subsection{Napovedni modeli}
\begin{itemize}
\item[1R] Moja implementacija 1R algoritma...
\item[RF] Random forest
\end{itemize}
\section{Rezultati}
Moje ime v tekmovalnem sistemu: Anže Pečar
\subsection{Rezultati oddaj}
\begin{table}[H]
\caption{Oddaje}
\begin{tabular}{ c c c c c l }
 Pred 12.3 & Ime metode & Oddaja & ocena mere F & ocena mere F na strežniku & Komentar\\
  \hline
  * & 1R & 12 & 0.36 & 0.36 & Najboljši rezultat metode \\
  \* & RF & 1 & 1 & 1 & Najboljši rezultat metode \\
 \end{tabular}
\end{table}
%\begin{figure}[H]
%\begin{center}
%\includegraphics[scale=0.2]{skupno100.png}
%\caption{Rezultati 100 permutacij za različne vrednosti Alpha}
%\label{skupno100}
%\end{center}
%\end{figure}

\subsection{Hitrost izvajanja}
500 RF:
\begin{verbatim}
real	889m59.265s
user	883m40.190s
sys	3m9.084s
\end{verbatim}

\section{Izjava o izdelavi domače naloge}
Domačo nalogo in pripadajoče programe sem izdelal sam.
\end{document}
