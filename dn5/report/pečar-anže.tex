% To je predloga za poročila o domačih nalogah pri predmetih, katerih
% nosilec je Blaž Zupan. Seveda lahko tudi dodaš kakšen nov, zanimiv
% in uporaben element, ki ga v tej predlogi (še) ni. Več o LaTeX-u izveš na
% spletu, na primer na http://tobi.oetiker.ch/lshort/lshort.pdf.
%
% To predlogo lahko spremeniš v PDF dokument s pomočjo programa
% pdflatex, ki je del standardne instalacije LaTeX programov.

\documentclass[a4paper,11pt]{article}
\usepackage{a4wide}
\usepackage{fullpage}
\usepackage[utf8x]{inputenc}
\usepackage[slovene]{babel}
\selectlanguage{slovene}
\usepackage[toc,page]{appendix}
\usepackage[pdftex]{graphicx} % za slike
\usepackage{setspace}
\usepackage{color}
\definecolor{light-gray}{gray}{0.95}
\usepackage{listings} % za vključevanje kode
\usepackage{hyperref}
\usepackage{float}
\usepackage{verbatim}
\renewcommand{\baselinestretch}{1.2} % za boljšo berljivost večji razmak
\renewcommand{\appendixpagename}{Priloge}

\lstset{ % nastavitve za izpis kode, sem lahko tudi kaj dodaš/spremeniš
language=Python,
basicstyle=\footnotesize,
basicstyle=\ttfamily\footnotesize\setstretch{1},
backgroundcolor=\color{light-gray},
}

\title{Peta domača naloga}
\author{Anže Pečar (63060257)}
\date{\today}

\begin{document}

\maketitle

\section{Uvod}

Cilj domače naloge je bil seznaniti se z linearno regresijo.
\section{Podatki}

Podatki sem pridobil...

\section{Rezultati}
\subsection{Prva to"cka}
\subsection{Druga to"cka}
\subsection{Tretja to"cka}

%\begin{table}[H]
%\caption{Oddaje}
%\begin{tabular}{ c | c | c | c | c | p{6cm} }
%  & Ime metode & Oddaja & ocena F &  F & Komentar\\
%  \hline \hline
%  * & 1R & 07.03. 13:31:56 & 0.33735 & 0.33878 & 1R s štetjem atributov \\ \hline
%  * & 1RS &09.03. 09:18:21 & 0.36952 & 0.36384 & 1R s seštevamjem vrednosti atributov \\\hline
%  * & RF &10.03. 09:44:08 & 0.38073 & 0.40977 & 250 dreves, prag 0.20, max 6 napovedanih razredov \\ \hline
%  * & RF &11.03. 08:09:49 & 0.37891 & 0.40387 & 500 dreves, prag 0.20, max 6 napovedanih razredov \\ \hline
%  & RFN &17.03. 16:02:10 & 0.38635 & 0.41439 & 500 dreves, normaliziran prag 0.5 \\ 
%\hline
%  & RFN &17.03. 23:29:35 & 0.38235 & 0.40337 & 500 dreves, normaliziran prag 0.1921 \\ \hline
%
%  & RFDP &18.03. 12:53:21 & 0.39655 & 0.43659 & 500 dreves, dinamični prag 0.221 \\ 
% \end{tabular}
% \label{tabela}
%\end{table}

%\begin{figure}[H]
%\begin{center}
%\includegraphics[scale=0.2]{skupno100.png}
%\caption{Rezultati 100 permutacij za različne vrednosti Alpha}
%\label{skupno100}
%\end{center}
%\end{figure}

\section{Izjava o izdelavi domače naloge}
Domačo nalogo in pripadajoče programe sem izdelal sam.


\begin{thebibliography}{9}

\bibitem{mining}
   Ian H. Witten \& Eibe Frank,
   \emph{Data Mining Practical Machine Learning Tools and Techniques, Second Edition}
   Morgan Kaufmann Publishers,  
   2005.

\end{thebibliography}

\end{document}
